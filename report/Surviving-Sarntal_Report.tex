% This is samplepaper.tex, a sample chapter demonstrating the
% LLNCS macro package for Springer Computer Science proceedings;
% Version 2.21 of 2022/01/12
%
\documentclass[a4paper,12pt]{llncs}
%
\usepackage{makeidx}  % allows for indexgeneration
\makeindex
%
%
\usepackage[utf8]{inputenc}
\usepackage[english]{babel}      % chnage the used langauge here!!!
\usepackage[T1]{fontenc}
%
% T1 fonts will be used to generate the final print and online PDFs,
% so please use T1 fonts in your manuscript whenever possible.
% Other font encondings may result in incorrect characters.
%
\usepackage{graphicx}
% correct colors for the hyperrefs
\usepackage[plainpages=false,hypertexnames=true,pdfnewwindow=true,backref=page,colorlinks=true,citecolor=blue,linkcolor=black,urlcolor=blue,filecolor=blue]{hyperref}
%
% additional (maybe useful) packages
\usepackage{amssymb}                % more math symbols
\usepackage{amsmath}                % better equations
\usepackage{booktabs}               % better tables
\usepackage[dvipsnames]{xcolor}     % more colors
\usepackage{spverbatim}             % verbatim with automatic line breaks
\usepackage{dsfont}                 % math fonts
\usepackage{csquotes}               % correct "" based on the used language
\usepackage{listings}               % source code
\usepackage{tikz}                   % custom (vector) graphics
\usetikzlibrary{shapes, arrows.meta, positioning, decorations.pathreplacing, trees, patterns, calligraphy}
%
% pseudo-code
\usepackage{algorithm} 
\usepackage{algpseudocode} 
\newcommand{\algorithmautorefname}{Algorithmus}
%
%
% page format ===============================================================
\hoffset=-1.25truecm
\setlength{\topmargin}{0.0cm}
\setlength{\textheight}{23.0cm}
\setlength{\footskip}{1.5cm}
\setlength{\textwidth}{15.4cm}
\setlength{\evensidemargin}{1.5cm}
\setlength{\oddsidemargin}{1.5cm}
\setlength{\parskip}{1ex}
\setlength{\parindent}{0pt}
\setlength{\marginparwidth}{1.4cm}
\setlength{\marginparsep}{1mm}

\pagestyle{plain}

% LstListing-Format ==========================================================
\lstdefinestyle{cpp}{
  language=C++,
  basicstyle=\small\ttfamily,
  frame=tb,
  xleftmargin=\parindent,
  keywordstyle=\color{blue},
  stringstyle=\color{red},
  commentstyle=\color{ForestGreen},
  framexleftmargin=5pt,
  framexrightmargin=5pt,
  framextopmargin=5pt,
  framexbottommargin=5pt,
  literate={~}{$\sim$}1
}

% macro definitions ==========================================================
% numbers -------------------------------------------------------------
\newcommand{\N}{{\mathbb{N}}}
\newcommand{\R}{{\mathbb{R}}}
\newcommand{\C}{{\mathbb{C}}}
\newcommand{\Z}{{\mathbb{Z}}}
\newcommand{\Q}{{\mathbb{Q}}}
\renewcommand{\O}{{\mathcal{O}}}
\newcommand\norm[1]{\lVert#1\rVert}
\renewcommand\qed{$\hfill\square$}
\newcommand\scprod[2]{\langle #1 , #2 \rangle}
\usepackage{wrapfig}
\usepackage{url}
\usepackage{xurl}
%
%
\def\myverzeichnis{.}
%
\numberwithin{equation}{section}
%
%
% images -----------------------------------------------------------------------
% #1 filename;  #2 Label;  #3 caption;  #4 short-caption
\newcommand{\image}[4]{%
  \begin{figure}[htbp]%
    \begin{center}%
      \includegraphics{#1}%
      \caption[#4]{#3}%
      \label{#2}%
    \end{center}%
  \end{figure}%
}

% image with a specific width -----------------------------------------------------------------
% #1 filename;  #2 width;  #3 Label;  #4 caption;  #5 short-caption
\newcommand{\imagewithwidth}[5]{
  \begin{figure}[htbp]%
    \begin{center}%
      \includegraphics[width=#2]{#1}%
      \caption[#5]{#4}%
      \label{#3}%
    \end{center}%
  \end{figure}
}

% ============================================================================
\begin{document}

% =========== Das war der Vorspann, jetzt geht's los! ========================

% ============================================================================
% =============  AB HIER DARF UND SOLL GETIPPT WERDEN ========================
% ============================================================================
%
\author{Bj\"orn Aheimer, Felix R\"ohr, Daniel Six, Aleksis Vezenkov, Anietta Weckauff}
\index{Bj\"orn Aheimer, Felix R\"ohr, Daniel Six, Aleksis Vezenkov, Anietta Weckauff}
%
% Das Institut wird fuer den Betreuer missbraucht ...
\institute{{\bf Betreuer:} Carme Homs-Pons}
\authorrunning{Bj\"orn Aheimer, Felix R\"ohr, Daniel Six, Aleksis Vezenkov, Anietta Weckauff}
\title{Towards real-time physics in a 2D survival-based game}
%
%
\maketitle              % typeset the header of the contribution
\thispagestyle{empty}
%
%
\begin{abstract}
  \textcolor{red}{\textbf{// TODO}}
  This project builds up on the outcome of a project that took place in the Ferienakademie 2023
  course \glqq{}Let's play! Simulated Physics for Games\grqq{} organised by Prof. Hans-Joachim Bungartz and
  Prof. Miriam Schulte. The participants of the course1 created the \glqq Surviving Sarntal\grqq{} game, the
  starting point of this project.
  \glqq Surviving Sarntal\grqq{} is a single-player survival-based game where the player is a hiker who goes
  up a mountain while avoiding rock-fall. The game has fun visuals, including a very unique kill-bar.
  The player can choose between keyboard input and game-pad input to interact with the game.
  The code is written in C++ and follows the entity component system (ECS) architectural pattern.
  The raylib library is used for the visuals.
\end{abstract}
%
% Introduction -----------------------------------------------------------------
\section{Introduction}

%
%
% Methodology ------------------------------------------------------------------
\section{Methodology}
\begin{itemize}
  \item Arbeitsweise: Scrum und andere Techniken
  \item Umgebung, Installation, \dots
  \item DevOps
  \item Quality Assurance
  \item Tranformation des alten Spiels
\end{itemize}

% Results ----------------------------------------------------------------------
\section{The Game}

\subsection{Goal of The Game}

\subsection{Entities}

\subsection{The Terrain}

\section{The Game Loop}

main Loop

\subsection{Input}

- short explanation of libs and how and when processed (maybe howto)


\subsection{Real-Time Rigid Body Physics}

\subsubsection{The Physics Loop}

\subsubsection{Spawning}

\subsubsection{Destructing}

\subsubsection{Entity Movement}

\subsubsection{Collision Detection}

\subsubsection{Collision Resolution}

%\vspace{-\abovedisplayskip}
\begin{definition}
  \textbf{Ein Beispiel Konzept} Lorem Ipsum dolor irgendwas.
  %Ein iteratives Verfahren heißt \textbf{konsistent}, falls die richtige Lösung nicht weiter transformiert wird.
  %Wenn das Verfahren für beliebige initiale Approximationen gegen die korrekte Lösung strebt, heißt es \textbf{konvergent}.
  %\vspace{-\belowdisplayskip}
\end{definition}
%\vspace{-\belowdisplayskip}
%Iterative Methoden bieten sich insbesondere an, wenn das zu lösende LGS Ergebnis einer Diskretisierungsaufgabe ist und die exakte Lösung selbst nur eine Annäherung an die zugrundeliegende Aufgabenstellung darstellt \cite{numLGS}.
Eine Citation sieht so aus Barrett et al. \cite{templates} zwei Arten von iterativen Methoden.

%\vspace{-\abovedisplayskip}
{\flushleft\begin{definition} \textbf{Ein Algorithmus (pls don't delete, I need that as ref)}\\
%\noindent\begin{minipage}{0.35\textwidth}
%  \begin{align}
%    r_{(0)} &= b - Ax_{(0)}\\ 
%    \alpha_{i} \notag&= \frac{\norm{r_{(i)}}_{2}^{2}}{\norm{r_{(i)}}_{A}^{2}} \\
%    &= \frac{r_{(i)}^{T}r_{(i)}}{r_{(i)}^{T}Ar_{(i)}}\\
%    x_{(i+1)} &= x_{(i)} + \alpha_{i}r_{(i)}\\
%    r_{(i+1)} &= b - Ax_{(i+1)}\label{eq:r}\\
%    &= r_{(i)} - \alpha_{i}Ar_{(i)}\label{eq:r_Alternativ}
%  \end{align}
%\end{minipage}
%\qquad
%\quad
%\begin{minipage}{0.55\textwidth}
%\vspace{-\abovedisplayskip}
\centering
\begin{minipage}{0.65\textwidth}
  \vspace{-\abovedisplayskip}
  \begin{algorithm}[H]
  \caption{The Steepest Descent Method}\label{alg:steep}
  \hspace*{\algorithmicindent} \textbf{Input:} $A\in\R^{n\times n}, \, b, x\in\R^{n}, \, i_{max}\in\N^{+}, \, \epsilon\in\R^{+}$\\
  \hspace*{\algorithmicindent} \textbf{Output:} $x\in\R^n, \, Ax = b' \text{ mit } \Delta b:=b'-b\leq\epsilon\norm{b}_{2}$
    \begin{algorithmic}[1]
    \Procedure{SteepestDescent}{$A, \, b, \, x, \, i_{max}, \, \epsilon$}
    \State $i \gets 0; \quad r \gets b - Ax; \quad \tau \gets r^{T}r; \quad stopCrit \gets \epsilon^{2}\norm{b}_{2}^{2}$ 
  %  \State $r \gets b - Ax$ 
  %  \State $\tau \gets r^{T}r$
  %  \State $stopCrit \gets \epsilon^{2}\norm{b}_{2}^{2}$
    \While{$(i < i_{max}$ \textbf{and} $\tau > stopCrit)$}
    \State $q \gets Ar; \quad \alpha \gets \frac{\tau}{r^{T}q}; \quad x \gets x + \alpha r$
  %  \State $\alpha \gets \frac{\tau}{r^{T}q}$
  %  \State $x \gets x + \alpha r$
    \If{$n>100$ \textbf{and} $i \text{\textbf{ mod }} \lfloor \sqrt{n} \rfloor = 0$}
      \State $r \gets b-Ax$\label{alg:steep:r}
    \Else 
      \State $r = r - \alpha q$\label{alg:steep:r_Alternativ}
    \EndIf
    \State $\tau \gets r^{T}r$
    \State $i \gets i+1$
    \EndWhile
    \State \Return $x$
    \EndProcedure
    \end{algorithmic}
  \end{algorithm}
\end{minipage}
%\vspace{-2\belowdisplayskip}
\end{definition}}
\vspace{\belowdisplayskip}

\begin{corollary}\label{cor:steep}
  Oh partigano, portami via.
\end{corollary}
%\vspace{-\belowdisplayskip}
%

\subsection{Rendering}

- short explanation of algorithms and libs 

%
% Conclusion -------------------------------------------------------------------
\section{Conclusion / Discussion}
%
% Future Work ------------------------------------------------------------------
\section{Future Work}
- open issues, Todos and ideas
%
% Appendix ---------------------------------------------------------------------
\appendix
%
\section{Angaben zur Reproduzierbarkeit usw.}
\label{Anhang}
Die verwendeten Software-Versionen (Python, NumPy, SciPy, ProbNum), weitere Plots sowie die Implementierungen finden sich in dem folgenden Git-Repository (Read-only access):
\url{https://gitfront.io/r/B-Aheimer/voAnN37ECCoS/Iterative-Loeser-fuer-LGS-Seminararbeit/}.
Für alle Randomisierungen wurde ein Random Number Generator mit festgelegtem Seed bereitgestellt.


% LaTeX-Tips für uns -----------------------------------------------------------
%
%NICHT LÖSCHEN!
%
%
%\subsection{Anmerkungen zur Einleitung}
%Hier kommt noch mehr Text. 
%Wir verweisen dazu auf \cite{thisdocument}.
%
%% Formulas -----------------------------------------------------------------
%Eine schöne Formel ist
%\[ u(\vec{x}) = \sum_{i=1}^N \alpha_i \varphi_i(\vec{x}) \,, \]
%aber das geht auch inline als $u(\vec{x}) = \sum_{i=1}^N \alpha_i \varphi_i(\vec{x})$, also mitten im Text.
%
%Es können auch Formeln dargstellt werden, die über mehrere Zeilen gehen, aber trotzdem schön zueinander ausgerichtet sind:
%\begin{align*}
%    f(x) &= a \cdot (1 + b) \\
%         &= a + ab
%\end{align*}
%
%% Images/Graphics -----------------------------------------------------------------
%Was noch fehlt ist ein Bild, z.B.\ das aus \autoref{fig:grid1} oder \autoref{fig:grid2}. 
%Wir können dazu prima die tollen Makros, die oben im Vorspann definiert wurden, verwenden.
%Beispielsweise mit folgenden Befehlen:
%\begin{spverbatim}
%\image{figures/grid_l2.png}{fig:grid1}{Dies ist ein sogenanntes dünnes Gitter zum Level 2.}{Die Kurzform lasse ich meistens leer}
%\imagewithwidth{figures/grid_l2.png}{2cm}{fig:grid2}{Dies ist ein sogenanntes dünnes Gitter zum Level 2 in 2cm Breite.}{}
%\end{spverbatim}
%
%Die Bilder werden automatisch nach vernünftigen Kriterien platziert, daher immer im Text mit \verb!\autoref{}! drauf verweisen (bei den Beispielen mit \verb!\autoref{fig:grid1}! und \verb!\autoref{fig:grid2}!).
%\image{figures/grid_l2.png}{fig:grid1}{Dies ist ein sogenanntes dünnes Gitter zum Level 2.}{Die Kurzform lasse ich meist leer}
%\imagewithwidth{figures/grid_l2.png}{2cm}{fig:grid2}{Dies ist ein sogenanntes dünnes Gitter zum Level 2 in 2cm Breite.}{}
%
%% Anmerkung: damit LaTeX nicht denkt, dass ein Punkt den Satz beendet
%% (da spendiert LaTeX gerne mehr Zwischenraum), können wir das
%% Leerzeichen mit Backslash als Leerzeichen markieren. Damit LaTeX
%% ein Leerzeichen setzt, bei dem es keinen Zeilenumbruch geben darf,
%% kann man die Tilde verwenden.
%% Tables -----------------------------------------------------------------
%Was wir hin und wieder noch brauchen ist eine Tabelle, wie z.B.\ \autoref{tab:irgendwas}.
%\begin{table}[htbp]
%  \centering
%  \caption{Diese Tabelle zeigt nicht die Daten von etwas Sinnvollem, sondern einfach irgend etwas. Tabellenbeschriftungen sind oft drüber.}
%  \label{tab:irgendwas}
%  \begin{tabular}{lrcp{5cm}}
%    \toprule
%    \multicolumn{3}{c}{Spalten} & Absatz 5cm \\
%    \cmidrule(lr){1-3}
%    linksbündig & rechtsbündig & zentriert & \\
%    \midrule
%    1.0 & -1.1 & 1.2 & toller Text, der nach 5cm umbricht und dafür brauchen wir einfach mehr Text. \\
%    4321.1 & 6543.2 & 7654.3 & mehr Text \\
%    2.44 & 4.66 & 6.88 & 8.00 \\
%    \bottomrule
%  \end{tabular}
%\end{table}
%
%
%% Source-Code -----------------------------------------------------------------
%\subsection{Quellcode}
%Code-Beispiele können mittels \texttt{lstlisting}-Environment eingebunden werden.
%Siehe \autoref{lst:mylisting} als Beispiel.
%Alternativen wie \texttt{minted} sind selbstverständlich auch erlaubt, solange sie Features wie Syntax-Highlighting und Zeilennummern mitbringen.
%Code-Beispiele sollten minimal sein, d.h.\ auf den Punkt gebracht und keinen überflüssigen Code beinhalten.
%Es muss standardkonformer Code sein und mit hinzugefügtem Boilerplate-Code (main, Auslassungen von Überflüssigem, \dots) ohne Fehler compilierbar sein.
%
%Quellcode aus Dateien kann per \texttt{lstinputlisting} einbezogen werden.
%Für Inline-Code \texttt{lstinline} verwenden.
%Für abstrakte Algorithmen (kein C++-Code) besser eines der algorithm-Packages verwenden.
%
%\begin{lstlisting}[style=cpp, caption={Example using Lstlisting}, label={lst:mylisting}, numbers=left]
%// I'm a comment!
%template <typename T>
%struct LessThan {
%  bool operator(T a, T b) { return a < b; };
%};
%
%/*
% * I'm a multiline comment!
% * Ich bin in Kommentar, der mehrere Zeilen verwendet!
% */
%std::vector<int> v = { 5, 4, 3, 2, 1 };
%std::sort(v.begin(), v.end(), LessThan<int>());
%
%std::cout << "Hello, World" << std::endl;
%\end{lstlisting}
%
%
%% Pseudo-Code -----------------------------------------------------------------
%\subsection{Pseudo-Code}
%Für Pseudo-Code kann die \texttt{algorithm}-Umgebung verwendet werden.
%Ein Beispiel ist in \autoref{alg:ppo} zu sehen.
%\begin{algorithm}
%	\caption{PPO}
%	\begin{algorithmic}[htbp]
%		\For {$iteration=1,2,\ldots$}
%			\For {$actor=1,2,\ldots,N$}
%				\State Run policy $\pi_{\theta_{old}}$ in environment for $T$ time steps
%				\State Compute advantage estimates $\hat{A}_{1},\ldots,\hat{A}_{T}$
%			\EndFor
%			\State Optimize surrogate $L$ wrt. $\theta$, with $K$ epochs and minibatch size $M\leq NT$
%			\State $\theta_{old}\leftarrow\theta$
%		\EndFor
%	\end{algorithmic} 
%	\label{alg:ppo}
%\end{algorithm}
%
%
%% Custom graphics -----------------------------------------------------------------
%\subsection{Graphiken}
%Wenn man einfache Graphiken in seiner Ausarbeitung verwenden will, bietet es sich heirfür immer an, diese selbst zu machen. 
%Eine Möglichkeit hierfür ist zum Beispiel TikZ.
%Aber Achtung: zu viele TikZ Bilder können die Compile-Zeit von LaTex negativ beeinflussen.
%
%\begin{figure}[htbp]
%    \centering
%    \begin{tikzpicture}
%        \fill[orange] (-0.5, -0.5) rectangle (2.5, 1.5);
%        \node[circle, fill=white, draw=black] (A) at (1, 1) {$A$};
%        \node[circle, fill=white, draw=black] (B) at (2, 0) {$B$};
%        \node[circle, fill=white, draw=black] (C) at (0, 0) {$C$};
%        \draw[-to] (A) -- (B);
%        \draw[-to] (B) -- (C);
%        \draw[-to] (C) -- (A);
%    \end{tikzpicture}
%    \caption{Ein einfaches Beispiel eines Graphen gezeichnet via TikZ.}
%    \label{fig:tikz}
%\end{figure}

%\subsection{Zum Schluss}
%\dots viel Spaß!


% ---- Bibliography ----
\bibliographystyle{splncs04}%alpha, arabic
\bibliography{bib/literatur.bib}

\end{document}
