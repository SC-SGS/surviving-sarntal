\section{Conclusion / Discussion}

The original Surviving Sarntal was developed during the Ferienakademie 2023 as a proof of concept under heavy time constraints, hence it suffered from issues with code maintainability, extensibility, and readability.
Therefore, we adapted the game from the flecs ECS framework \cite{flecs_library} to OOP, which required significant refactoring, including redesigning the game loop.
Nonetheless, the new architecture, while less performant, is now far more modular and maintainable, easing future development.

Designing our own physics engine was another pivotal aspect of the rework. 
The initial implementation handled simple collisions with basic circular objects and a piecewise linear terrain.
Our custom physics engine supports arbitrary convex polygons, which allowed us to introduce richer game mechanics and interactions. 
Though this increased the complexity of collision detection algorithms, our performance benchmarks show the engine is capable of handling a sufficiently large number of entities without significant performance degradation. 
However, future versions may require optimization for handling larger, more dynamic environments.

In addition to technical improvements, we reworked the terrain system by incorporating two-dimensional splines, offering a more challenging and visually engaging landscape. 
This change, coupled with the introduction of new mechanics like terrain fall-off and item interaction, has drastically enhanced the gameplay experience. 
The new audio and graphics have further transformed the original idea for the game into a well-rounded game. 

In conclusion, the rework of Surviving Sarntal resulted in a more modular, maintainable, and engaging game. 
We applied core software engineering principles and best practices throughout the development process, ensuring high code quality while vastly improving user experience. 
