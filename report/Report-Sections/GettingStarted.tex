% \subsection{Way of working: Scrum and other methodologies}
% Prior to starting the project, we, as a team, devised a project charter that encompasses our vision and mission for the upcoming project.
% Additionally, it includes an analysis of potential challenges and strategies to address them,
% as well as concrete working agreements, including tools to use, style guides and organisational decisions.
% The further development of the game and our collaboration within the team were based on the aforementioned project charter.

% On a weekly basis, a meeting was held with our supervisor.
% During these meetings, the current progress regarding the project timeline, ideas and challenges occurring in development were discussed.
% Furthermore, two time slots were determined each week for a brief discussion of each team member's current work.
% The GitLab \glqq Issues\grqq{} were used to determine feasible work packages, which then could be assigned to one team member.
% Additionally, the Gitlab \glqq Issue Board\grqq{} was used to provide an overview of the progress of each issue.

% Once an issue had been resolved by one team member, a comprehensive review of another team member was required
% before the changes could be incorporated into the mainline.
% This measure was implemented with the objective of guaranteeing code quality and correctness.

% The project's continuous progress was ensured by structuring it in sprints.
% We carried out a total of three sprints, whereby at the beginning of each sprint we defined the objectives at the end of the sprint
% and the issues to be dealt with.

\section{Getting Started}
The project is compatible with Ubuntu 24.04.1 and also supports macOS. 
It includes several bash scripts to simplify installation, running, and testing. 
A platform-agnostic script handles installing dependencies across different operating systems. 
There are separate installation processes for playing the game and for development, with an additional script provided for developers. 
Information on contributing to development can be found in the project documentation \cite{ProjectWiki}.

\subsection{Installing and Running the Game}
To run the game, the script \code{user\_setup.sh} can be used to facilitate the installation of the required packages and dependencies.
In that case, run the following command from the root project directory:

\begin{verbatim}
$ source build-utils/user_setup.sh
\end{verbatim}

%The following packages will be installed based on the operating system:
%\begin{itemize}
%    \item \textbf{Ubuntu:}
%  \code{build-essential cmake xorg-dev libsdl2-dev}
%    \item \textbf{MacOS:}
%  \code{cmake sdl2}.
%\end{itemize}
%The installation of the packages on MacOS requires the use of Homebrew.
%If you do not have it installed you can still run the \code{user\_setup.sh} script, which will install it.

The installation script handles dependencies for both operating systems.
Once all of the dependencies have been installed, the script \code{run.sh} can be executed in order to run the game:
\begin{verbatim}
$ source build-utils/run.sh
\end{verbatim}

\subsection{YAML Configuration}

In this project, a YAML configuration file is used to manage settings and parameters that control various aspects of the game, such as behavior, resources and environmental constants.
The human-readable YAML format allows easy modification of game elements, without the need for recompilation.
This approach enhances development workflow by supporting rapid iterations and testing. The configuration includes:

\begin{itemize}
  \item \textbf{Run Mode and Game Settings:} 
  The configuration begins with a global setting for development mode (\code{run-dev-mode}). 
  When enabled, it provides additional functionality for developers, helping to debug specific components.

  \item \textbf{Assets:} 
  Specifies all audio and visual assets for the game.

  \item \textbf{Items:} 
  Configures game items with their attributes.

  \item \textbf{Landmarks:}
  Defines landmarks, such as the Statue of Liberty or Mount Everest, which act as height references during gameplay.

  \item \textbf{Game Constants:} 
  Includes parameters for in-game mechanics like player movement, health, and environmental interactions. Constants for terrain, item spawning, and physics can be adjusted here, allowing developers to fine-tune gameplay by modifying values such as jump velocity or rock spawn frequency.
\end{itemize}

\subsection{Development Mode}
A dedicated \code{DevMode} is provided, allowing developers to test the game in a controlled environment. This mode bypasses the standard game flow to aid in debugging and fine-tuning specific mechanics.
\pagebreak